\documentclass[a4paper]{article}

%% Language and font encodings
\usepackage[english]{babel}
\usepackage[utf8x]{inputenc}
\usepackage[T1]{fontenc}

%% Sets page size and margins
\usepackage[a4paper,top=3cm,bottom=2cm,left=3cm,right=3cm,marginparwidth=1.75cm]{geometry}

%% Useful packages
\usepackage{amsmath}
\usepackage{graphicx}
\usepackage[colorinlistoftodos]{todonotes}
\usepackage[colorlinks=true, allcolors=blue]{hyperref}

\title{Odi Cricket Prediciton}
\author{Aditya Yadav\\ 
\texttt{adiyadav.1729@gmail.com }\and
Aniket Dhawad\\
\texttt{aniketdhawadiitg1@gmail.com} \and
Vikash Goyal \\
\texttt{vikashgoyal2701@gmail.com}}



\begin{document}
\maketitle

\begin{abstract}
The aim of this study was to investigate to what degree a machine learning model can capture outcome of match depending on team's composition and strength. The target competition was the World Cup 2019.   Basic features comprising of batting and bowling metrics was used along with features capturing the impact and quality of players. Team's batting and bowling strength was created based on playing squad for a match and venue where the match was played. 
The model was tested with different training dataset based on timeline. Result showed how greatly odi crickets have changed with time. 
\end{abstract}

\section{Introduction}

One Day International(ODI) Cricket is one of the three formats which are played at International Level. International Cricket Council(ICC), governing body of cricket, has  currently 12  Full Team Members and 92 Associate Members. 
ODI cricket started in 1971 as second form of cricket after test cricket to cater entertainment to larger set of spectators.In this format, both teams are allowed to play for fixed overs and the one with more runs is declared winner. Originally started with 60 overs and white jerseys, ODI cricket has gone through various transition over years  
to stay relevant and consistently entertaning. 50 overs , powerplay overs, field restrictions, batting friendly grounds are among few significant changes that has played role in its evolution.

Cricket World Cup, the most important tournament in this format generally played every 4 years, with preliminary qualification rounds leading up to a finals tournament. The tournament is one of the world's most viewed sporting events and is considered the "flagship event of the international cricket calendar" by the ICC.  World cup 2019 has 10 teams participating. In league round all 10 teams would be playing against each other. Top 4 would march into semi- finals. Winner of the semi-finalist would be playing for the coveted title. There would be 45 league matches, 2 semi-finals and 1 finals. This paper aimed at predicting the result of 48 matches.

We collected data from cricinfo and relianceiccranking. Over 100 features were created which included players and teams features as well.  Pearson correlation, chi square test and recursive feature elimination were used for feature elimination. The selected features were used as inputs to three different classification algorithms: logistic regression,
random forest, boosted decision trees. Ensemble of weak classifiers were also tested. Best performing model was selected as final one to predict the outcome.


\section{Methodolgy}

\subsection{Data}

Data was scrapped from www.espncricinfo.com. Firstly, list of all matches played was scrapped. Secondly, list of all players played for different country was scrapped. Note that data had match\_id, player\_id which was key in getting different information
at match and player level respectively. We used www.relianceiccrankings.com to collect information of player batting and bowling ranking at the start of every month from 1971 till date. 

Therefore, for every match we had batting and bowling scorecard, player details at match level and their ranking. 



\subsection{Feature Engineering}
\subsubsection{
Player Level Features
}

Player Batting Average, Strike rate, Bowling Average , Bowling Economy, Bowling Strike Rate 

Impact Feature

Form factor/Quality Feature


Batting:
While comparing batsmen , generally batting average is used a metric to identify the batsmen in form. Though, it is reasonable to comparable batsmen on average when a considerable time period is taken into account. However, it does not give clear picture of batsmen form in last few( say 10) matches. For instance, a batsmen with score sequence  120,0,30,40,10  another with score 60,45,45,35 have batting average of 50. 

Bowling:
This metric was calculated to capture the form and consistency of the bowler in the recent few matches before the actual match. Bowler's economy, strike rate and average in the last 10 matches was taken into consideration while calculating this metric. 
Example:
Show data of last 10 matches of a particular inform bowler and a out of form bowler.




\subsubsection{Team Level features
}

Team Strength:
For every team, batting and bowling strength was calculated by considering the ratings of players of the team as per the ICC world rankings. The batting strength was calculated by taking the average rating of top 6 batsmen and bowling strength by taking the average rating of top 5 bowlers. We assumed the minimum rating  for player who was not present in the top 100 list.

Pitch Weighted Team Strength:
In order to take into the impact of the pitch, we tried to quantify whether the pitch was a high scoring or a low scoring one. We did this by calculating the average runs that are scored on a particular pitch. If the pitch was high scoring one, we increased the batting strength of a team by a particular factor and vice versa.

Our thought behind this feature was that flat pitches give advantage to batsmen and reduce the advantage to bowlers.


\subsection{Modelling}
 
 Feature Selection
 
Use the table and tabular commands for basic tables --- see Table~\ref{tab:widgets}, for example. 

\begin{table}
\centering
\begin{tabular}{l|r}
Item & Quantity \\\hline
Widgets & 42 \\
Gadgets & 13
\end{tabular}
\caption{\label{tab:widgets}An example table.}
\end{table}

\subsection{Improvement }

\LaTeX{} is great at typesetting mathematics. Let $X_1, X_2, \ldots, X_n$ be a sequence of independent and identically distributed random variables with $\text{E}[X_i] = \mu$ and $\text{Var}[X_i] = \sigma^2 < \infty$, and let
\[S_n = \frac{X_1 + X_2 + \cdots + X_n}{n}
      = \frac{1}{n}\sum_{i}^{n} X_i\]
denote their mean. Then as $n$ approaches infinity, the random variables $\sqrt{n}(S_n - \mu)$ converge in distribution to a normal $\mathcal{N}(0, \sigma^2)$.


\subsection{How to create Sections and Subsections}

Use section and subsections to organize your document. Simply use the section and subsection buttons in the toolbar to create them, and we'll handle all the formatting and numbering automatically.

\subsection{How to add Lists}

You can make lists with automatic numbering \dots

\begin{enumerate}
\item Like this,
\item and like this.
\end{enumerate}
\dots or bullet points \dots
\begin{itemize}
\item Like this,
\item and like this.
\end{itemize}

\subsection{How to add Citations and a References List}

You can upload a \verb|.bib| file containing your BibTeX entries, created with JabRef; or import your \href{https://www.overleaf.com/blog/184}{Mendeley}, CiteULike or Zotero library as a \verb|.bib| file. You can then cite entries from it, like this: \cite{greenwade93}. Just remember to specify a bibliography style, as well as the filename of the \verb|.bib|.

You can find a \href{https://www.overleaf.com/help/97-how-to-include-a-bibliography-using-bibtex}{video tutorial here} to learn more about BibTeX.

We hope you find Overleaf useful, and please let us know if you have any feedback using the help menu above --- or use the contact form at \url{https://www.overleaf.com/contact}!

\bibliographystyle{alpha}
\bibliography{sample}

\end{document}